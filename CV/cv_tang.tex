%%%%%%%%%%%%%%%%%%%%%%%%%%%%%%%%%%%%%%%%%
% Wilson Resume/CV
% XeLaTeX Template
% Version 1.0 (22/1/2015)
%
% This template has been downloaded from:
% http://www.LaTeXTemplates.com
%
% Original author:
% Howard Wilson (https://github.com/watsonbox/cv_template_2004) with
% extensive modifications by Vel (vel@latextemplates.com)
%
% License:
% CC BY-NC-SA 3.0 (http://creativecommons.org/licenses/by-nc-sa/3.0/)
%
%%%%%%%%%%%%%%%%%%%%%%%%%%%%%%%%%%%%%%%%%

%----------------------------------------------------------------------------------------
%	PACKAGES AND OTHER DOCUMENT CONFIGURATIONS
%----------------------------------------------------------------------------------------

\documentclass[10pt]{article} % Default font size

%%%%%%%%%%%%%%%%%%%%%%%%%%%%%%%%%%%%%%%%%
% Wilson Resume/CV
% Structure Specification File
% Version 1.0 (22/1/2015)
%
% This file has been downloaded from:
% http://www.LaTeXTemplates.com
%
% License:
% CC BY-NC-SA 3.0 (http://creativecommons.org/licenses/by-nc-sa/3.0/)
%
%%%%%%%%%%%%%%%%%%%%%%%%%%%%%%%%%%%%%%%%%

%----------------------------------------------------------------------------------------
%	PACKAGES AND OTHER DOCUMENT CONFIGURATIONS
%----------------------------------------------------------------------------------------

\usepackage[a4paper, hmargin=25mm, vmargin=30mm, top=20mm]{geometry} % Use A4 paper and set margins

\usepackage{fancyhdr} % Customize the header and footer

\usepackage{lastpage} % Required for calculating the number of pages in the document

\usepackage{hyperref} % Colors for links, text and headings

\setcounter{secnumdepth}{0} % Suppress section numbering

\usepackage{comment}

%\usepackage[proportional,scaled=1.064]{erewhon} % Use the Erewhon font
%\usepackage[erewhon,vvarbb,bigdelims]{newtxmath} % Use the Erewhon font
%\usepackage[utf8]{inputenc} % Required for inputting international characters
\usepackage[T1]{fontenc} % Output font encoding for international characters

\usepackage{fontspec} % Required for specification of custom fonts
\setmainfont[Path = ./fonts/,
Extension = .otf,
BoldFont = Erewhon-Bold,
ItalicFont = Erewhon-Italic,
BoldItalicFont = Erewhon-BoldItalic,
SmallCapsFeatures = {Letters = SmallCaps}
]{Erewhon-Regular}

\usepackage{color} % Required for custom colors
\definecolor{slateblue}{rgb}{0.17,0.22,0.34}
\definecolor{maroon}{RGB}{128,0,0}

\usepackage{sectsty} % Allows customization of titles
\sectionfont{\color{slateblue}} % Color section titles

\fancypagestyle{plain}{\fancyhf{}\cfoot{\thepage\ }} % Define a custom page style
\pagestyle{plain} % Use the custom page style through the document
\renewcommand{\headrulewidth}{0pt} % Disable the default header rule
\renewcommand{\footrulewidth}{0pt} % Disable the default footer rule

\setlength\parindent{0pt} % Stop paragraph indentation

% Non-indenting itemize
\newenvironment{itemize-noindent}
{\setlength{\leftmargini}{0em}\begin{itemize}}
{\end{itemize}}

% Text width for tabbing environments
\newlength{\smallertextwidth}
\setlength{\smallertextwidth}{\textwidth}
\addtolength{\smallertextwidth}{-2cm}

\newcommand{\sqbullet}{~\vrule height 1ex width .8ex depth -.2ex} % Custom square bullet point definition
\newcommand\emptybullet[1][1.2ex]{\hspace*{#1}}


%----------------------------------------------------------------------------------------
%	MAIN HEADER COMMAND
%----------------------------------------------------------------------------------------

\renewcommand{\title}[1]{
{\huge{\color{slateblue}\textbf{#1}}}\\ % Header section name and color
\rule{\textwidth}{0.5mm}\\ % Rule under the header
}

%----------------------------------------------------------------------------------------
%	JOB COMMAND
%----------------------------------------------------------------------------------------

\newcommand{\job}[6]{
\begin{tabbing}
\hspace{2.5cm} \= \kill
\textbf{#1} \> \href{#4}{#3} \\
\textbf{#2} \>\+ \textit{#5} \\
\begin{minipage}{\smallertextwidth}
\vspace{1mm}
#6
\end{minipage}
\end{tabbing}
\vspace{1mm}
}

%----------------------------------------------------------------------------------------
%	SKILL GROUP COMMAND
%----------------------------------------------------------------------------------------

\newcommand{\skillgroup}[2]{
\begin{tabbing}
\hspace{5mm} \= \kill
\sqbullet \>\+ \textbf{#1} \\
\begin{minipage}{\smallertextwidth}
\vspace{2mm}
#2
\end{minipage}
\end{tabbing}
}

%----------------------------------------------------------------------------------------
%	INTERESTS GROUP COMMAND
%-----------------------------------------------------------------------------------------

\newcommand{\pubgroup}[1]{
\begin{tabbing}
\hspace{1cm} \= \kill
#1
\end{tabbing}
}

\newcommand{\pub}[1]{\sqbullet \> #1\\} % Define a custom command for individual interests

\newcounter{pubcounter}
\setcounter{pubcounter}{0}
\newcommand\showpubcounter{\stepcounter{pubcounter}\thepubcounter}
\newcommand{\pubWithNum}[1]{P\showpubcounter. \> #1\\} % Define a custom command for individual interests
\newcounter{mancounter}
\setcounter{mancounter}{0}
\newcommand\showmancounter{\stepcounter{mancounter}\themancounter}
\newcommand{\manWithNum}[1]{M\showmancounter. \> #1\\} % Define a custom command for individual interests

%----------------------------------------------------------------------------------------
%	TABBED BLOCK COMMAND
%----------------------------------------------------------------------------------------

\newcommand{\tabbedblock}[1]{
\begin{tabbing}
\hspace{2.5cm} \= \hspace{1cm} \= \kill
#1
\end{tabbing}
}


%----------------------------------------------------------------------------------------
%	PROJECT COMMAND
%----------------------------------------------------------------------------------------

\newcommand\tab[1][1cm]{\hspace*{#1}}

\newenvironment{myindentpar}[1]%
  {\begin{list}{}%
          {\setlength{\leftmargin}{#1}}%
          \item[]%
  }
  {\end{list}
}

\newcommand{\mentor}[3]{
\sqbullet \tab \textbf{#1} \hspace*{\fill} #2 \\

\vspace{-6mm}
\begin{myindentpar}{1.2cm}
#3
\end{myindentpar}
}

\newcommand{\project}[3]{
\sqbullet \tab \textbf{#1} \hspace*{\fill} #2

\vspace{-2mm}
\begin{myindentpar}{1.2cm}
#3
\end{myindentpar}
}

\newcounter{rescounter}
\setcounter{rescounter}{0}
\newcommand\showrescounter{\stepcounter{rescounter}\therescounter}
\newcommand{\projectWithNum}[3]{
R\showrescounter. $\>$ \textbf{#1} \hspace*{\fill} #2

\vspace{-2mm}
\begin{myindentpar}{0.72cm}
#3
\end{myindentpar}
}

\newcommand{\projectWithNumIndent}[3]{
R\showrescounter. $\>$ \textbf{#1} \hspace*{\fill} #2

\vspace{-2mm}
\begin{myindentpar}{0.9cm}
#3
\end{myindentpar}
}

\newcommand{\projectWithNumUrl}[4]{
R\showrescounter. $\>$ \textbf{#1} \hspace*{\fill} #2 \\
\phantom{R1. }$\>$ Code: \url{#3}

\vspace{-2mm}
\begin{myindentpar}{0.72cm}
#4
\end{myindentpar}
}

\newcommand{\system}[2]{
\sqbullet \tab \textbf{#1}

\vspace{-2mm}
\begin{myindentpar}{1.2cm}
#2
\end{myindentpar}
}

\newcommand{\intern}[5]{
\sqbullet $\>$ \textbf{#1} \hspace*{\fill} #2 \\
\emptybullet $\>$ Project: #3 \hspace*{\fill} Mentor: #4

\vspace{-2mm}
\begin{myindentpar}{0.36cm}
#5
\end{myindentpar}
}






 % Include the file specifying document layout

%----------------------------------------------------------------------------------------

\begin{document}

%----------------------------------------------------------------------------------------
%	NAME AND CONTACT INFORMATION
%----------------------------------------------------------------------------------------

\title{Dixin Tang} % Print the main header

%------------------------------------------------

\parbox{0.5\textwidth}{ % First block
\begin{tabbing} % Enables tabbing
\hspace{2.5cm} \= \hspace{4cm} \= \kill % Spacing within the block
{\bf Address} \> 5730 S Ellis Ave, JCL 299 \\ % Address line 1
\> Chicago, IL 60637 \\ % Address line 2
\end{tabbing}}
\hfill % Horizontal space between the two blocks
\parbox{0.5\textwidth}{ % Second block
\begin{tabbing} % Enables tabbing
\hspace{2.5cm} \= \hspace{4cm} \= \kill % Spacing within the block
{\bf Homepage} \> \href{http://people.cs.uchicago.edu/~totemtang}{\url{people.cs.uchicago.edu/~totemtang}} \\ 
{\bf Email} \> totemtang@uchicago.edu \\ % Email address 
\end{tabbing}}
\vspace{-10mm}
%----------------------------------------------------------------------------------------
%	PERSONAL PROFILE
%----------------------------------------------------------------------------------------

\section{Research Areas}

Query Processing/Optimization, Cloud Database, Transaction Processing

\vspace{-5mm}
%----------------------------------------------------------------------------------------
%	EDUCATION SECTION
%----------------------------------------------------------------------------------------

\section{Education}

\tabbedblock{
    {2015-present} \> Ph.D. Candidate in Computer Science - The University of Chicago \\[5pt]
    \> Advisor: Aaron Elmore \\[5pt]
    {2011-2014} \> M.S. in Computer Science - Institute of Computing Technology, Chinese Academy of Sciences \\[5pt]
    \> Advisor: Wei Li \\[5pt]
    {2007-2011} \> B.S. in Software Engineering - Huazhong University of Science \& Technology 
}

\vspace{-5mm}

%----------------------------------------------------------------------------------------
%	Publication SECTION
%----------------------------------------------------------------------------------------
\section{Publications at UChicago}
\pubgroup{
    \pub{CrocodileDB in Action: Resource-Efficient Query Execution by Exploiting Time Slackness}
       \> \textbf{Dixin Tang}, Zechao Shang, Aaron J. Elmore, Sanjay Krishnan, Michael J. Franklin \\
       \> \textbf{VLDB 2020, Demo} \\[5mm]

    \pub{Thrifty Query Execution via Incrementability}
       \> \textbf{Dixin Tang}, Zechao Shang, Aaron J. Elmore, Sanjay Krishnan, Michael J. Franklin \\
       \> \textbf{SIGMOD 2020} \\[5mm]

    \pub{CrocodileDB: Efficient Database Execution through Intelligent Deferment}
    \> Zechao Shang, Xi Liang, \textbf{Dixin Tang}, Cong Ding, Aaron J. Elmore, Sanjay Krishnan, Michael J. Franklin \\
    \> \textbf{CIDR 2020} \\[5mm]


    \pub{Intermittent Query Processing}
       \> \textbf{Dixin Tang}, Zechao Shang, Aaron J. Elmore, Sanjay Krishnan, Michael J. Franklin \\
       \> \textbf{VLDB 2019} \\[5mm]

    \pub{Socrates: The New SQL Server in the Cloud}
       \> Panagiotis Antonopoulos, Alex Budovski, Cristian Diaconu, Alejandro Hernandez Saenz, Jack Hu, \\ 
	   \> Hanuma Kodavalla, Donald Kossmann, Umar Farooq Minhas, Naveen Prakash, Hugh Qu,  \\ 
	   \> Chaitanya Sreenivas Ravella, Krystyna Reisteter, Sheetal Shroti, \textbf{Dixin Tang}, Vikram Wakade \\
       \> \textbf{SIGMOD 2019} \\[5mm]

    \pub{Toward Coordination-free and Reconfigurable Mixed Concurrency Control}
       \> \textbf{Dixin Tang}, Aaron J. Elmore \\
       \> \textbf{USENIX'ATC 2018} \\[5mm]

    \pub{Adaptive Concurrency Control: Despite the Looking Glass, One Concurrency Control Does Not Fit All}
       \> \textbf{Dixin Tang}, Hao Jiang, Aaron J. Elmore \\
       \> \textbf{CIDR 2017}
}

\vspace{-5mm}


\section{Earlier Publications}
\pubgroup{
    \pub{A Case Study of Optimizing Big Data Analytical Stacks Using Structured Data Shuffling}
       \> \textbf{Dixin Tang}, Taoying Liu, Rubao Lee, Hong Liu, Wei Li \\
       \> \textbf{BigData Congress 2016} \\[5mm]

    \pub{SparkArray: An Array-Based Scientific Data Management System Built on Apache Spark}
       \> Wenjuan Wang, Taoying Liu, \textbf{Dixin Tang}, Hong Liu, Wei Li, Rubao Lee \\
       \> \textbf{NAS 2016} \\[5mm]

    \pub{A Case Study of Optimizing Big Data Analytical Stacks Using Structured Data Shuffling}
       \> \textbf{Dixin Tang}, Taoying Liu, Rubao Lee, Hong Liu, Wei Li \\
       \> \textbf{CLUSTER 2015, Short Paper} \\[5mm]   

    \pub{RHJoin: A Fast and Space-efficient Join Method for Log Processing in MapReduce}
       \> \textbf{Dixin Tang}, Taoying Liu, Hong Liu, Wei Li \\
       \> \textbf{ICPADS 2014} \\[5mm]

    \pub{Optimizing the Join Operation on Hive to Accelerate Cross-Matching in Astronomy}
       \> Liang Li, \textbf{Dixin Tang}, Taoying Liu, Hong Liu, Wei Li, Chenzhou Cui  \\
       \> \textbf{IPDPS Workshops 2014}
}

\vspace{-5mm}

\section{Industrial Experience}
\intern{Internship at Microsoft Research}{June 2018-Sep. 2018}
{Benchmarking Socrates}{Umar Farooq Minhas}
{Socrates is a new cloud-native database that decouples computation from storage. 
My internship job is to test the new database architecture of Socrates 
in the industrial setting, understand its performance bottlenecks, 
and propose optimization opportunities.}

\vspace{-5mm}

%----------------------------------------------------------------------------------------
% Research Projects
%----------------------------------------------------------------------------------------
\section{Research Projects at UChicago}

\project{CrocodileDB: Resource-efficient Database Execution}{July 2019-Present}
{CrocodileDB is a new database for processing dynamic datasets (e.g. streaming tuples). 
It exploits intelligent deferment to achieve low resource consumption 
and high query performance at the same time. 
In CrocodileDB, users can trade-off between resource consumption and query performance 
by specifying a maximally allowed time slackness 
between the data is ready and the corresponding query result is returned. 
Our system integrates the slackness constraints along with information 
about new data and query structures into the underlying query optimizer 
to unlock new resource-efficient query execution plans.}

\project{Incrementability-aware Query Processing}{Mar. 2019-Present}
{This project studies how to efficiently maintain non-positive standing queries. 
While incremental executions can reduce query latency, it increases total query work (i.e. CPU cycles) 
when we eagerly maintain non-positive queries 
because output tuples in earlier executions are repeatedly removed by later executions. 
To reduce the total work of incremental executions without sacrificing query latency, 
we selectively defer some parts of a query that are less amenable to incremental execution. 
To enable this optimization, we define a metric, \textit{incrementability} that 
quantifies the cost-effectiveness of incremental executions. 
Our system can leverage this metric to decide how eagerly and lazily we should execute different parts of a query.}

\project{Intermittent Query Processing}{Dec. 2017-Sep. 2019}
{This project studies how to maintain a standing query 
over an incomplete data set with the remaining data
arriving in an intermittent yet predictable way. 
To process this data arrival pattern, 
we propose \textit{Intermittent Query Processing}(IQP) 
that does not keep the query active all the time but releases some memory resources when there is no data to process. 
By exploiting the knowledge of incoming data, 
IQP selectively keeps a subset of intermediate states when the query is inactive. 
Therefore, with limited memory consumption IQP can achieve low latency of updating query results for new data.}

\project{Adaptive Concurrency Control for Main-memory Database}{Sep. 2015-Nov. 2017}
{
We build a main-memory database that supports adaptively mixing multiple forms of concurrency control with minimal overhead. 
Our system can decompose the workload into partitions and 
selects a concurrency control protocol for each partition of workload that the protocol is optimized for, 
and during workload changes adaptively reconfigure the protocols online.
}

\section{Earlier Projects}
\project{Structured Data Shuffling for Big Data Analytical Stacks}{Nov. 2013-Jan. 2015}
{
We build a structured data shuffling procedure that can leverage the semantics of SQL queries to apply efficient compression algorithms 
and discard unnecessary data during data shuffling. 
}

\project{A Fast and Space-efficient Join Method for Log Processing in MapReduce}{Sep. 2012-Nov. 2013}
{
We design a join method that achieves high query performance with a small extra storage cost for log processing. 
It shuffles the log table to avoid huge storage consumption and optimizes the shuffle procedure to achieve high query performance.
}

\vspace{-5mm}

%----------------------------------------------------------------------------------------
%	SYSTEM BUILDING EXPERIENCE
%----------------------------------------------------------------------------------------

\section{System Building Experience}

\system{Multi-Query Optimization in SparkSQL}
{We extend SparkSQL to support multi-query optimization for all TPC-H queries. 
Our framework takes Dataframe queries as input and generates a query plan 
that shares the execution of all submitted queries. 
We enhance each intermediate tuple during query execution to 
additionally include the query IDs that the tuple is valid to 
and modify the SparkSQL operator to leverage this information 
to support shared query execution.}

\system{Complex Materialized View Maintenance in SparkSQL}
{We extend SparkSQL to support maintaining materialized views with deletes and updates.
Our system supports complex materialized views that are beyond select-project-join-aggregate queries. 
Specifically, the operators our system supports include select, project, aggregate, sort, materialize, distinct, 
and all kinds of joins (inner, outer, and anti join) and there is no ordering limit about these operators.}

\system{Intermediate State Management in PostgreSQL}
{We extend PostgreSQL to support materialized view maintenance with intermediate state management. 
Our modified PostgreSQL can selectively discard some intermediate states to save memory resources. 
When a discarded state is required to process new data, we can rebuild this state from saved intermediate states 
to quickly update the materialized view.
}

\system{Adaptive Concurrency Control in Main-Memory Database}
{We develop a key-value main-memory database which hosts several concurrency control protocols 
including partition-based concurrency control, two-phase locking, and optimistic concurrency control protocols. 
Our database uses a machine learning model to adaptively apply a protocol to a partition of the database 
and reconfigures protocols in response to workload changes.}


%----------------------------------------------------------------------------------------
%	AWARD SECTION
%----------------------------------------------------------------------------------------

\section{Honors \& Awards}

\tabbedblock{
    {2018} \> USENIX ATC'18 Student Travel Grant \\[5pt]
    {2016} \> University Unrestricted (UU) Fellowship - The University of Chicago \\[5pt]
    {2016} \> CERES 1st year Graduate Research Award - The University of Chicago
}
\vspace{-5mm}


%----------------------------------------------------------------------------------------
%	EMPLOYMENT HISTORY SECTION
%----------------------------------------------------------------------------------------

\section{Teaching Assistant}

\tabbedblock{
    {Fall 2015} \> MPCS 51040 - C programming  \\[5pt]
    {Spring 2016} \> MPCS 52040 - Distributed Systems  \\[5pt]
    {Winter 2017} \> CMSC 23500 - Introduction to Database \\[5pt]
    {Winter 2018} \> CMSC 23500 - Introduction to Database  \\[5pt]
    {Winter 2019} \> CMSC 23500 - Introduction to Database  \\[5pt]
    {Winter 2020} \> CMSC 23500 - Introduction to Database  
}
\vspace{-5mm}

%\job
%{Sep 2011 -}{Present}
%{Lehman Brothers, 1234 Mario Park, San Francisco, CA, United States}
%{http://www.lehmanbrothers.com}
%{Senior Developer / Technical Team Lead}
%{Lorem ipsum dolor sit amet, consectetur adipiscing elit. Duis elementum nec dolor sed sagittis. Cras justo lorem, volutpat mattis lacus vel, consequat aliquam quam. Interdum et malesuada fames ac ante ipsum primis in faucibus.\\
%\rule{0mm}{5mm}\textbf{Technologies:} Ruby on Rails 2.3, Amazon EC2, NoSQL data stores, memcached, collaborative matching, Facebook Graph API.}

%------------------------------------------------

%----------------------------------------------------------------------------------------
%	REFEREE SECTION
%----------------------------------------------------------------------------------------

\section{Referees}

\parbox{0.5\textwidth}{ % First block
\begin{tabbing}
\hspace{1.5cm} \= \hspace{4cm} \= \kill % Spacing within the block
{\bf Name} \> Aaron Elmore \\ % Referee name
{\bf Affiliate} \> University of Chicago \\ % Referee company
{\bf Position} \> Assistant Professor \\ % Referee job title 
{\bf Contact} \> aelmore@cs.uchicago.edu % Referee contact information
\end{tabbing}}
\\
\parbox{0.5\textwidth}{ % Second block
\begin{tabbing}
\hspace{1.5cm} \= \hspace{4cm} \= \kill % Spacing within the block
{\bf Name} \> Michael Franklin \\ % Referee name
{\bf Affiliate} \> University of Chicago \\ % Referee company
{\bf Position} \> Liew Family Chairman of Computer Science \\ % Referee job title 
{\bf Contact} \> mjfranklin@uchicago.edu % Referee contact information
\end{tabbing}}
\\
\parbox{0.5\textwidth}{ % First block
\begin{tabbing}
\hspace{1.5cm} \= \hspace{4cm} \= \kill % Spacing within the block
{\bf Name} \> Sanjay Krishnan \\ % Referee name
{\bf Affiliate} \> University of Chicago \\ % Referee company
{\bf Position} \> Assistant Professor \\ % Referee job title 
{\bf Contact} \> skr@cs.uchicago.edu % Referee contact information
\end{tabbing}}
\\
\parbox{0.5\textwidth}{ % Second block
\begin{tabbing}
\hspace{1.5cm} \= \hspace{4cm} \= \kill % Spacing within the block
{\bf Name} \> Umar Farooq Minhas \\ % Referee name
{\bf Affiliate} \> Microsoft Research \\ % Referee company
{\bf Position} \> Principle Researcher \\ % Referee job title 
{\bf Contact} \> ufminhas@microsoft.com % Referee contact information
\end{tabbing}}
\\
\parbox{0.5\textwidth}{ % First block
\begin{tabbing}
\hspace{1.5cm} \= \hspace{4cm} \= \kill % Spacing within the block
{\bf Name} \> Wei Li\\ % Referee name
{\bf Affiliate} \> Institute of Computing Technology \\ % Referee company
{\bf Position} \> Associate Professor \\ % Referee job title 
{\bf Contact} \> liwei@ict.ac.cn % Referee contact information
\end{tabbing}}
\hfill


%----------------------------------------------------------------------------------------

\end{document}
