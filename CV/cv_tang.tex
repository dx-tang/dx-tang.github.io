%%%%%%%%%%%%%%%%%%%%%%%%%%%%%%%%%%%%%%%%%
% Wilson Resume/CV
% XeLaTeX Template
% Version 1.0 (22/1/2015)
%
% This template has been downloaded from:
% http://www.LaTeXTemplates.com
%
% Original author:
% Howard Wilson (https://github.com/watsonbox/cv_template_2004) with
% extensive modifications by Vel (vel@latextemplates.com)
%
% License:
% CC BY-NC-SA 3.0 (http://creativecommons.org/licenses/by-nc-sa/3.0/)
%
%%%%%%%%%%%%%%%%%%%%%%%%%%%%%%%%%%%%%%%%%

%----------------------------------------------------------------------------------------
%	PACKAGES AND OTHER DOCUMENT CONFIGURATIONS
%----------------------------------------------------------------------------------------

\documentclass[10pt]{article} % Default font size

%%%%%%%%%%%%%%%%%%%%%%%%%%%%%%%%%%%%%%%%%
% Wilson Resume/CV
% Structure Specification File
% Version 1.0 (22/1/2015)
%
% This file has been downloaded from:
% http://www.LaTeXTemplates.com
%
% License:
% CC BY-NC-SA 3.0 (http://creativecommons.org/licenses/by-nc-sa/3.0/)
%
%%%%%%%%%%%%%%%%%%%%%%%%%%%%%%%%%%%%%%%%%

%----------------------------------------------------------------------------------------
%	PACKAGES AND OTHER DOCUMENT CONFIGURATIONS
%----------------------------------------------------------------------------------------

\usepackage[a4paper, hmargin=25mm, vmargin=30mm, top=20mm]{geometry} % Use A4 paper and set margins

\usepackage{fancyhdr} % Customize the header and footer

\usepackage{lastpage} % Required for calculating the number of pages in the document

\usepackage{hyperref} % Colors for links, text and headings

\setcounter{secnumdepth}{0} % Suppress section numbering

\usepackage{comment}

%\usepackage[proportional,scaled=1.064]{erewhon} % Use the Erewhon font
%\usepackage[erewhon,vvarbb,bigdelims]{newtxmath} % Use the Erewhon font
%\usepackage[utf8]{inputenc} % Required for inputting international characters
\usepackage[T1]{fontenc} % Output font encoding for international characters

\usepackage{fontspec} % Required for specification of custom fonts
\setmainfont[Path = ./fonts/,
Extension = .otf,
BoldFont = Erewhon-Bold,
ItalicFont = Erewhon-Italic,
BoldItalicFont = Erewhon-BoldItalic,
SmallCapsFeatures = {Letters = SmallCaps}
]{Erewhon-Regular}

\usepackage{color} % Required for custom colors
\definecolor{slateblue}{rgb}{0.17,0.22,0.34}
\definecolor{maroon}{RGB}{128,0,0}

\usepackage{sectsty} % Allows customization of titles
\sectionfont{\color{slateblue}} % Color section titles

\fancypagestyle{plain}{\fancyhf{}\cfoot{\thepage\ }} % Define a custom page style
\pagestyle{plain} % Use the custom page style through the document
\renewcommand{\headrulewidth}{0pt} % Disable the default header rule
\renewcommand{\footrulewidth}{0pt} % Disable the default footer rule

\setlength\parindent{0pt} % Stop paragraph indentation

% Non-indenting itemize
\newenvironment{itemize-noindent}
{\setlength{\leftmargini}{0em}\begin{itemize}}
{\end{itemize}}

% Text width for tabbing environments
\newlength{\smallertextwidth}
\setlength{\smallertextwidth}{\textwidth}
\addtolength{\smallertextwidth}{-2cm}

\newcommand{\sqbullet}{~\vrule height 1ex width .8ex depth -.2ex} % Custom square bullet point definition
\newcommand\emptybullet[1][1.2ex]{\hspace*{#1}}


%----------------------------------------------------------------------------------------
%	MAIN HEADER COMMAND
%----------------------------------------------------------------------------------------

\renewcommand{\title}[1]{
{\huge{\color{slateblue}\textbf{#1}}}\\ % Header section name and color
\rule{\textwidth}{0.5mm}\\ % Rule under the header
}

%----------------------------------------------------------------------------------------
%	JOB COMMAND
%----------------------------------------------------------------------------------------

\newcommand{\job}[6]{
\begin{tabbing}
\hspace{2.5cm} \= \kill
\textbf{#1} \> \href{#4}{#3} \\
\textbf{#2} \>\+ \textit{#5} \\
\begin{minipage}{\smallertextwidth}
\vspace{1mm}
#6
\end{minipage}
\end{tabbing}
\vspace{1mm}
}

%----------------------------------------------------------------------------------------
%	SKILL GROUP COMMAND
%----------------------------------------------------------------------------------------

\newcommand{\skillgroup}[2]{
\begin{tabbing}
\hspace{5mm} \= \kill
\sqbullet \>\+ \textbf{#1} \\
\begin{minipage}{\smallertextwidth}
\vspace{2mm}
#2
\end{minipage}
\end{tabbing}
}

%----------------------------------------------------------------------------------------
%	INTERESTS GROUP COMMAND
%-----------------------------------------------------------------------------------------

\newcommand{\pubgroup}[1]{
\begin{tabbing}
\hspace{1cm} \= \kill
#1
\end{tabbing}
}

\newcommand{\pub}[1]{\sqbullet \> #1\\} % Define a custom command for individual interests

\newcounter{pubcounter}
\setcounter{pubcounter}{0}
\newcommand\showpubcounter{\stepcounter{pubcounter}\thepubcounter}
\newcommand{\pubWithNum}[1]{P\showpubcounter. \> #1\\} % Define a custom command for individual interests
\newcounter{mancounter}
\setcounter{mancounter}{0}
\newcommand\showmancounter{\stepcounter{mancounter}\themancounter}
\newcommand{\manWithNum}[1]{M\showmancounter. \> #1\\} % Define a custom command for individual interests

%----------------------------------------------------------------------------------------
%	TABBED BLOCK COMMAND
%----------------------------------------------------------------------------------------

\newcommand{\tabbedblock}[1]{
\begin{tabbing}
\hspace{2.5cm} \= \hspace{1cm} \= \kill
#1
\end{tabbing}
}


%----------------------------------------------------------------------------------------
%	PROJECT COMMAND
%----------------------------------------------------------------------------------------

\newcommand\tab[1][1cm]{\hspace*{#1}}

\newenvironment{myindentpar}[1]%
  {\begin{list}{}%
          {\setlength{\leftmargin}{#1}}%
          \item[]%
  }
  {\end{list}
}

\newcommand{\mentor}[3]{
\sqbullet \tab \textbf{#1} \hspace*{\fill} #2 \\

\vspace{-6mm}
\begin{myindentpar}{1.2cm}
#3
\end{myindentpar}
}

\newcommand{\project}[3]{
\sqbullet \tab \textbf{#1} \hspace*{\fill} #2

\vspace{-2mm}
\begin{myindentpar}{1.2cm}
#3
\end{myindentpar}
}

\newcounter{rescounter}
\setcounter{rescounter}{0}
\newcommand\showrescounter{\stepcounter{rescounter}\therescounter}
\newcommand{\projectWithNum}[3]{
R\showrescounter. $\>$ \textbf{#1} \hspace*{\fill} #2

\vspace{-2mm}
\begin{myindentpar}{0.72cm}
#3
\end{myindentpar}
}

\newcommand{\projectWithNumIndent}[3]{
R\showrescounter. $\>$ \textbf{#1} \hspace*{\fill} #2

\vspace{-2mm}
\begin{myindentpar}{0.9cm}
#3
\end{myindentpar}
}

\newcommand{\projectWithNumUrl}[4]{
R\showrescounter. $\>$ \textbf{#1} \hspace*{\fill} #2 \\
\phantom{R1. }$\>$ Code: \url{#3}

\vspace{-2mm}
\begin{myindentpar}{0.72cm}
#4
\end{myindentpar}
}

\newcommand{\system}[2]{
\sqbullet \tab \textbf{#1}

\vspace{-2mm}
\begin{myindentpar}{1.2cm}
#2
\end{myindentpar}
}

\newcommand{\intern}[5]{
\sqbullet $\>$ \textbf{#1} \hspace*{\fill} #2 \\
\emptybullet $\>$ Project: #3 \hspace*{\fill} Mentor: #4

\vspace{-2mm}
\begin{myindentpar}{0.36cm}
#5
\end{myindentpar}
}






 % Include the file specifying document layout

\usepackage{xspace}

\newcommand{\db}{CrocodileDB\xspace}
\newcommand{\pg}{PostgreSQL\xspace}
\newcommand{\inqp}{InQP\xspace}
\newcommand{\iqp}{IQP\xspace}
\newcommand{\incability}{incrementability\xspace}
\newcommand{\Incability}{Incrementability\xspace}

\newcommand{\form}[1]{\texttt{\small #1}}

%----------------------------------------------------------------------------------------

\begin{document}

%----------------------------------------------------------------------------------------
%	NAME AND CONTACT INFORMATION
%----------------------------------------------------------------------------------------

\title{Dixin Tang {\small dixin [at] utexas [dot] edu}} % Print the main header
%\title{Dixin Tang} % Print the main header
\vspace{-5mm}

\begin{comment}
%------------------------------------------------
\parbox{0.5\textwidth}{ % First block
\begin{tabbing} % Enables tabbing
\hspace{2.5cm} \= \hspace{4cm} \= \kill % Spacing within the block
Postdoctoral Scholar \\
University of California, Berkeley \\
387 Soda Hall, Berkeley, CA 94720
\end{tabbing}}
\hfill % Horizontal space between the two blocks
\parbox{0.5\textwidth}{ % Second block
\begin{tabbing} % Enables tabbing
\hspace{2.5cm} \= \hspace{4cm} \= \kill % Spacing within the block
(+1) 510-365-9300 \\
totemtang@berkeley.edu \\
\url{https://people.eecs.berkeley.edu/~totemtang/}
\end{tabbing}}
\vspace{-5mm}
\end{comment}

%----------------------------------------------------------------------------------------
%	PERSONAL PROFILE
%----------------------------------------------------------------------------------------

\section{Research Areas}

Transaction Processing, Large-Scale Data Analysis, Data Systems for Machine Learning 

\vspace{-5mm}

\section{Employment}

\tabbedblock{
    {2024-present} \> Assistant Professor - University of Texas, Austin \\[5pt]
    {2021-2023} \> Postdoctoral Scholar - University of California, Berkeley \\[5pt]
    \> Advisor: Aditya G. Parameswaran, Associate Professor
}

\vspace{-5mm}

%----------------------------------------------------------------------------------------
%	EDUCATION SECTION
%----------------------------------------------------------------------------------------

\section{Education}

\tabbedblock{
    {2015-2020} \> Ph.D. in Computer Science - University of Chicago \\[5pt]
    \> Advisor: Aaron J. Elmore, Associate Professor \\[5pt]
    {2011-2014} \> M.S. in Computer Science - Institute of Computing Technology, Chinese Academy of Sciences \\[5pt]
    \> Advisor: Wei Li, Associate Professor \\[5pt]
    {2007-2011} \> B.S. in Software Engineering - Huazhong University of Science \& Technology 
}

\vspace{-5mm}

\begin{comment}
%----------------------------------------------------------------------------------------
%	Manuscript SECTION
%----------------------------------------------------------------------------------------

\section{Manuscripts}
\pubgroup{   	
    \manWithNum{FormS: A Python Library for Scalable Spreadsheet Formula Execution}
    	\> \textbf{Dixin Tang}*, Zixuan Yi*, Connor Lien, Ryan Sun, Aditya G. Parameswaran \\
    	\> \textbf{In Preparation} (*Equal contribution) \\[5mm]
    	
    \manWithNum{ShiftXplain: A Data Shift Explanation Framework}
    	\> Yong Wang, Shreya Shankar, \textbf{Dixin Tang}, Guoliang Li, Aditya G. Parameswaran \\
    	\> \textbf{In Preparation}
}
\vspace{-5mm}

\end{comment}

%----------------------------------------------------------------------------------------
%	Publication SECTION
%----------------------------------------------------------------------------------------
\section{Publications}
\pubgroup{
    \pubWithNum{Transactional Panorama: A Conceptual Framework for User Perception in Analytical Visual \\ \>   Interfaces (extended version)}
        \> \textbf{Dixin Tang}, Alan Fekete, Indranil Gupta, Aditya G. Parameswaran \\
        \> \textbf{VLDBJ 2025} (The \textbf{“Best of VLDB 2023”} special issue of VLDB Journal) \\[5mm]

    \pubWithNum{Tigon: A Distributed Database for a CXL Pod}
        \> Yibo Huang, Haowei Chen, Newton Ni, Yan Sun, Vijay Chidambaram, \textbf{Dixin Tang}, Emmett Witchel\\
    	\> \textbf{OSDI 2025} \\[5mm]

    \pubWithNum{Impeller: Stream Processing on Shared Logs}
        \> Zhiting Zhu, Zhipeng Jia, Newton Ni, \textbf{Dixin Tang}, Emmett Witchel \\
    	\> \textbf{EuroSys 2025} \\[5mm]

    \pubWithNum{Pasha: An Efficient, Scalable Database Architecture for CXL Pods}
        \> Yibo Huang, Newton Ni, Vijay Chidambaram, Emmett Witchel, \textbf{Dixin Tang} \\
    	\> \textbf{CIDR 2025} \\[5mm]

    \pubWithNum{Dealing with Acronyms, Abbreviations, and Typos in Real-World Entity Matching}
        \>  Joshua Wu, \textbf{Dixin Tang}, Nithin Chalapathi, Tristan Chambers, Julie Ciccolini, Cheryl Phillips, \\
        \> Lisa Pickoff-White, Aditya G. Parameswaran \\
    	\> \textbf{VLDB 2024, Industrial Track} \\[5mm]

    \pubWithNum{Visualizing Spreadsheet Formula Graphs Compactly}
    	\> Fanchao Chen, \textbf{Dixin Tang}, Haotian Li, Aditya G. Parameswaran \\
    	\> \textbf{VLDB 2023, Demo} \\[5mm]

    \pubWithNum{Transactional Panorama: A Conceptual Framework for User Perception in Analytical Visual Interfaces}
    	\> \textbf{Dixin Tang}, Alan Fekete, Indranil Gupta, Aditya G. Parameswaran \\
        \> \textbf{VLDB 2023} (\textbf{“Best of VLDB 2023”}) \\[5mm]
    	
    \pubWithNum{Efficient and Compact Spreadsheet Formula Graphs}
    	\> \textbf{Dixin Tang}, Fanchao Chen, Christopher De Leon, Tana Wattanawaroon, Jeaseok Yun, \\ 
    	\> Srinivasan Seshadri, Aditya G. Parameswaran \\
    	\> \textbf{ICDE 2023} \\[5mm]

    \pubWithNum{Flexible Rule-Based Decomposition and Metadata Independence in Modin: A Parallel Dataframe System}
    	\> Devin Petersohn*, \textbf{Dixin Tang}*, Rehan Durrani, Areg Melik-Adamyan, Joseph E. Gonzalez, \\
    	\> Anthony D. Joseph, Aditya G. Parameswaran \\
    	\> \textbf{VLDB 2022} (*Equal contribution) \\[5mm]

    \pubWithNum{Lux: Always-on Visualization Recommendations for Exploratory Dataframe Workflows}
    	\> Doris Jung-Lin Lee, \textbf{Dixin Tang}, Kunal Agarwal, Thyne Boonmark, Caitlyn Chen, Jake Kang, \\
    	\> Ujjaini Mukhopadhyay, Jerry Song, Micah Yong, Marti A. Hearst, Aditya G. Parameswaran \\
    	\> \textbf{VLDB 2022} \\[5mm]

    \pubWithNum{Enhancing the Interactivity of Dataframe Queries by Leveraging Think Time}
    	\> Doris Xin, Devin Petersohn, \textbf{Dixin Tang}, Yifan Wu, Joseph E. Gonzalez, Joseph M. Hellerstein,  \\
    	\> Anthony D. Joseph, Aditya G. Parameswaran \\
    	\> \textbf{IEEE Data Eng. Bull. 2021} \\[5mm]

    \pubWithNum{Resource-Efficient Shared Query Execution via Exploiting Time Slackness}
    	\> \textbf{Dixin Tang}, Zechao Shang, William Ma, Aaron J. Elmore, Sanjay Krishnan \\
    	\> \textbf{SIGMOD 2021} \\[5mm]

    \pubWithNum{CIAO: An Optimization Framework for Client-Assisted Data Loading}
    	\> Cong Ding, \textbf{Dixin Tang}, Xi Liang, Aaron J. Elmore, Sanjay Krishnan \\
    	\> \textbf{ICDE 2021, Short Paper} \\[5mm]
    	
    \pubWithNum{\db in Action: Resource-Efficient Query Execution by Exploiting Time Slackness}
       \> \textbf{Dixin Tang}, Zechao Shang, Aaron J. Elmore, Sanjay Krishnan, Michael J. Franklin \\
       \> \textbf{VLDB 2020, Demo} \\[5mm]

    \pubWithNum{Thrifty Query Execution via \Incability}
       \> \textbf{Dixin Tang}, Zechao Shang, Aaron J. Elmore, Sanjay Krishnan, Michael J. Franklin \\
       \> \textbf{SIGMOD 2020} \\[5mm]

    \pubWithNum{\db: Efficient Database Execution through Intelligent Deferment}
    \> Zechao Shang, Xi Liang, \textbf{Dixin Tang}, Cong Ding, Aaron J. Elmore, Sanjay Krishnan, Michael J. Franklin \\
    \> \textbf{CIDR 2020} \\[5mm]


    \pubWithNum{Intermittent Query Processing}
       \> \textbf{Dixin Tang}, Zechao Shang, Aaron J. Elmore, Sanjay Krishnan, Michael J. Franklin \\
       \> \textbf{VLDB 2019} \\[5mm]

    \pubWithNum{Socrates: The New SQL Server in the Cloud}
       \> Panagiotis Antonopoulos, Alex Budovski, Cristian Diaconu, Alejandro Hernandez Saenz, Jack Hu, \\ 
	   \> Hanuma Kodavalla, Donald Kossmann, Umar Farooq Minhas, Naveen Prakash, Hugh Qu,  \\ 
	   \> Chaitanya Sreenivas Ravella, Krystyna Reisteter, Sheetal Shroti, \textbf{Dixin Tang}, Vikram Wakade \\
       \> \textbf{SIGMOD 2019} \\[5mm]

    \pubWithNum{Toward Coordination-Free and Reconfigurable Mixed Concurrency Control}
       \> \textbf{Dixin Tang}, Aaron J. Elmore \\
       \> \textbf{USENIX'ATC 2018} \\[5mm]

    \pubWithNum{Adaptive Concurrency Control: Despite the Looking Glass, One Concurrency Control Does Not Fit All}
       \> \textbf{Dixin Tang}, Hao Jiang, Aaron J. Elmore \\
       \> \textbf{CIDR 2017}
}

\vspace{-5mm}


%\section{Earlier Publications (Before Ph.D.)} 
%\pubgroup{
%    \pubWithNum{A Case Study of Optimizing Big Data Analytical Stacks Using Structured Data Shuffling}
%       \> \textbf{Dixin Tang}, Taoying Liu, Rubao Lee, Hong Liu, Wei Li \\
%       \> \textbf{BigData Congress 2016} \\[5mm]
%
%    \pubWithNum{SparkArray: An Array-Based Scientific Data Management System Built on Apache Spark}
%       \> Wenjuan Wang, Taoying Liu, \textbf{Dixin Tang}, Hong Liu, Wei Li, Rubao Lee \\
%       \> \textbf{NAS 2016} \\[5mm]
%
%    \pubWithNum{A Case Study of Optimizing Big Data Analytical Stacks Using Structured Data Shuffling}
%       \> \textbf{Dixin Tang}, Taoying Liu, Rubao Lee, Hong Liu, Wei Li \\
%       \> \textbf{CLUSTER 2015, Short Paper} \\[5mm]   
%
%    \pubWithNum{RHJoin: A Fast and Space-Efficient Join Method for Log Processing in MapReduce}
%       \> \textbf{Dixin Tang}, Taoying Liu, Hong Liu, Wei Li \\
%       \> \textbf{ICPADS 2014} \\[5mm]
%
%    \pubWithNum{Optimizing the Join Operation on Hive to Accelerate Cross-Matching in Astronomy}
%       \> Liang Li, \textbf{Dixin Tang}, Taoying Liu, Hong Liu, Wei Li, Chenzhou Cui  \\
%       \> \textbf{IPDPS Workshops 2014}
%}

%\vspace{-5mm}

%----------------------------------------------------------------------------------------
%	SYSTEM BUILDING EXPERIENCE
%----------------------------------------------------------------------------------------

%\section{System Building Experience}
%
%\system{Multi-Query Optimization in SparkSQL}
%{I extend SparkSQL to support multi-query optimization for the full TPC-H test suite. 
%The framework takes Dataframe queries as input and generates a query plan 
%that shares the execution of all submitted queries. 
%I enhance each intermediate tuple during query execution to 
%additionally include the query IDs that the tuple is valid to 
%and modify SparkSQL operators to leverage this information 
%to support shared query execution.}
%
%\system{Complex Materialized View Maintenance in SparkSQL}
%{I extend SparkSQL to support maintaining materialized views with deletes and updates.
%The system supports complex materialized views that are beyond select-project-join-aggregate queries. 
%Specifically, the supported operators include select, project, aggregate, sort, materialize, distinct, 
%and joins (inner, outer, and anti join), and there is no ordering limit about these operators.}
%
%\system{Intermediate State Management in \pg}
%{I extend \pg to support materialized view maintenance with intermediate state management. 
%The modified \pg can selectively discard some intermediate states to save memory resources. 
%When a discarded state is required to process new data, the system can rebuild this state from saved intermediate states 
%to quickly update the materialized view.
%}
%
%\system{Adaptive Concurrency Control in Main-Memory Database}
%{I develop a key-value main-memory database which hosts several concurrency control protocols 
%including partition-based concurrency control, two-phase locking, and optimistic concurrency control protocols. 
%The database uses a machine learning model to adaptively apply a protocol to a partition of the database 
%and reconfigures protocols in response to workload changes.}
%
%
%\vspace{-5mm}

%\section{Mentoring Experience}

%\mentor{Cong Ding, Undergraduate}{July 2019-Present}
%{I mentor Cong Ding on client-assisted data loading project, 
%which exploits partial data loading to make data quickly available, but not heavily sacrifice query performance. 
%The research paper is under preparation.}

%\vspace{-5mm}

%----------------------------------------------------------------------------------------
%       SERVICE SECTION
%----------------------------------------------------------------------------------------
\section{Professional Services}
\tabbedblock{
    Program Committee: \\
    \tab SIGMOD - 2022, 2023, 2025, 2026 \\
    \tab SIGMOD Demo Track - 2022, 2024, 2025 \\
    \tab VLDB - 2025, 2026 \\ 
    \tab ICDE - 2025, 2026 \\ 
    \tab EDBT - 2024 \\ 
    \tab HILDA - 2023, 2024 \\ [5pt]

    Conference Reviewer: \\
    \tab IEEE VIS - 2021 \\ [5pt]

    Journal Reviewer: \\
    \tab VLDB Journal \\
    \tab Distributed and Parallel Databases Journal
}
\vspace{-5mm}

%----------------------------------------------------------------------------------------
% Mentoring Experience
%----------------------------------------------------------------------------------------
\begin{comment}
\section{Mentoring Experiences}

\vspace{-3mm}

\begin{table}[!h]
\hskip-0.2cm
\small
\begin{tabular}{lllll}
\textbf{Mentee} & \textbf{Affiliation}              & \textbf{Period}     & \textbf{Next Appointment} & \textbf{Authorship}   \\
Zixuan Yi     & Undergrad at Tsinghua University & Feb. 2022-Now       &                           & Co-first author of M1 \\
Fanchao Chen  & Undergrad at Fudan University    & Jun. 2022-Now       &                           & Second author of P2   \\
Connor Lien   & Undergrad at UC Berkeley         & Jan. 2022-Now       &                           & Co-author of M1       \\
Ryan Sun      & Undergrad at UC Berkeley         & Sep. 2022-Now       &                           & Co-author of M1       \\
Avinash Rao   & Undergrad at UC Berkeley         & Jan. 2022-Now       &                           &                       \\
Joshua Wu     & Master's student at UC Berkeley  & Jan. 2022-Now       &                           &                       \\
Todd Yu       & Master's student at UC Berkeley  & Sep. 2021-Now       &                           &                       \\
Christina Fan & Undergrad at UC Berkeley         & Sep. 2021-Now       &                           &                       \\
Kunal Agarwal & Master's student at UC Berkeley  & Jun. 2021-Jun. 2022 & SDE at Stripe         	& Co-author of P4       \\
Jonathan Yun  & Undergrad at UC Berkeley         & Mar. 2021-Jun. 2022 & SDE at Oracle             & Co-author of P2       \\
Chris De Leon & Master's student at UC Berkeley  & Mar. 2021-Jun. 2022 & SDE at Anchain.AI         & Co-author of P2       \\
William Ma    & Master's student at UChicago     & Jun. 2020-Oct. 2020 & SDE at Apple              & Co-author of P6       \\
Cong Ding     & Undergrad at Peking University   & Jun. 2019-Oct. 2020 & PhD at UW-Madison         & First author of P7   
\end{tabular}
\end{table}

\vspace{-5mm}
\end{comment}


%----------------------------------------------------------------------------------------
%	AWARD SECTION
%----------------------------------------------------------------------------------------

\begin{comment}
\section{Honors \& Awards}

\tabbedblock{
    {2018} \> USENIX ATC'18 Student Travel Grant \\[5pt]
    {2016} \> University Unrestricted (UU) Fellowship - The University of Chicago \\[5pt]
    {2016} \> CERES 1st year Graduate Research Award - The University of Chicago
}
\vspace{-5mm}

\section{Teaching Assistants}

\tabbedblock{
    {Winter 2020} \> CMSC 23500 - Introduction to Database Systems \\[5pt] 
    {Winter 2019} \> CMSC 23500 - Introduction to Database Systems \\[5pt] 
    {Winter 2018} \> CMSC 23500 - Introduction to Database Systems \\[5pt] 
    {Winter 2017} \> CMSC 23500 - Introduction to Database Systems \\[5pt] 
    {Spring 2016} \> MPCS 52040 - Distributed Systems  \\[5pt]
    {Fall 2015} \> MPCS 51040 - C Programming
}
\vspace{-5mm}
\end{comment}

\section{Teaching}

\tabbedblock{
    University of Texas at Austin: \\[3pt]
    {Instructor} \> CS 347 - Data Management, Fall 2024\\[5pt]
    {Instructor} \> CS 395T - Advanced Query Optimizations, Spring 2024\\[10pt]

    University of Chicago: \\[3pt]
    {TA} \> CMSC 23500 - Introduction to Database Systems, Winter 2017-2020 \\[5pt] 
    {TA} \> MPCS 52040 - Distributed Systems, Spring 2016 \\[5pt]
    {TA} \> MPCS 51040 - C Programming, Fall 2015
}
\vspace{-5mm}


\section{Industry Experience}
\intern{Internship at Microsoft Research}{June 2018-Sep. 2018}
{Benchmarking Socrates}{Umar Farooq Minhas}
{Socrates is a new cloud-native database that decouples computation from storage. 
My internship involved testing the new database architecture of Socrates 
in an industrial setting, understanding its performance bottlenecks, 
and proposing optimization opportunities.}

\vspace{-5mm}

\begin{comment}
%----------------------------------------------------------------------------------------
% Research Projects
%----------------------------------------------------------------------------------------
\section{Recent Research Projects (Detailed Descriptions)}

\projectWithNum{Data Shift Explaination}{Sep. 2021-Present}
{ShiftXplain is a framework for explaining data shift. 
Data shift is ubiquitous in real-world datasets 
due to the natural evolution in the underlying data 
relationships and patterns, often leading to a degradation 
of the performance of data-dependent applications (e.g., an ML model). 
ShiftXplain explains the data shift 
between two datasets using a conjunction of predicates 
and proposes a novel metric, shiftIndex, to capture 
both general and unique shift patterns. 
More importantly, shiftIndex is partially monotonic and bounded, 
which is leveraged by our search algorithm to efficiently 
prune the search space without sacrificing the explanation quality.}
 
\vspace{3mm}

\projectWithNumUrl{FormS: A Python Library for Scalable Spreadsheet Formula Execution}{Mar. 2022-Jun. 2023}
{https://github.com/forms-org/forms}
{FormS is a Python library for executing spreadsheet formulae 
in a distributed execution framework. 
In FormS, users write a list of formulae 
via a formula template based on autofill rules, 
as widely supported by today's spreadsheet systems. 
For example, applying a template \form{SUM(A1:A2,B\$1:B2)} 
to a column will generate $[\form{SUM(A1:A2,B\$1:B2)}, 
\form{SUM(A2:A3,B\$1:B3)}, \cdots]$, where the first range 
is generated similar to a sliding window while the 
second one is similar to an expanding window in databases. 
While a formula template is similar to a window operator in databases, 
efficiently executing a list of formulae is challenging 
due to two unique semantics in spreadsheets: 
1) each spreadsheet function may accept multiple windows with different sizes and types; 
2) spreadsheets support many functions that are not optimized by 
databases (e.g., \form{SUMIF} function) and require 
new optimizations to execute them efficiently. 
To address these challenges, I proposed novel logical and physical 
rewriting rules for executing a list of formulae 
efficiently in parallel. 
FormS now has support for over 50 popular spreadsheet functions; 
I am mentoring a few undergrads to support more functions 
for a public release. }

\projectWithNumUrl{Transactional Panorama: Enhancing User Perception in Analytical Interfaces}{Sep. 2021-Jun. 2023}
{https://github.com/transactional-panorama/TP}
{I developed transactional panorama, 
a conceptual framework for user perception when the visualizations 
in a visual analytical interface (e.g., a dashboard or spreadsheet) 
are being refreshed. In such an interface, it is common 
for users to modify the source data or filters 
to explore different visualization results. 
With large datasets, it takes a long time to 
refresh the visualization results 
while users continue to explore the results simultaneously. 
In this context, existing tools either (i) hide away
results that haven't been updated, hindering exploration;
(ii) make the updated results immediately available to the user
(on the same screen as old results), leading to confusion 
and incorrect insights; or (iii) present old---and therefore 
stale---results to the user during the update. 
I developed transactional panorama to 
discover new options for users 
and help users make appropriate trade-offs between 
the properties guaranteed for the visual results 
and the performance for presenting the updated results. 
Transactional panorama adopts database transactions 
to jointly model the system refreshing the visualization results 
and the user interacting with them, 
and considers three properties that are important for 
user perception: visibility (allowing users to continuously 
explore results), consistency (ensuring that results 
presented are from the same version of the data or filters), 
and monotonicity (making sure that results 
don't ``go back in time''). 
I characterized all feasible property combinations, 
formally proved their relative orderings for various 
performance criteria (e.g., the total time when 
the user sees stale results), and discussed their use cases. 
With transactional panorama, the user can explore 
visual results with desired properties guaranteed 
while the visual interface is being refreshed.}

\projectWithNumUrl{Taco: Efficient and Compact Spreadsheet Formula Graphs}{Jan. 2021-Jun. 2023}
{https://github.com/taco-org/taco}
{Taco is a framework for efficiently compressing, querying, 
and maintaining spreadsheet formula graphs to reduce the response time of spreadsheets. 
A formula graph is adopted to track the dependencies across spreadsheet formulae. 
When a spreadsheet cell is modified, the spreadsheet system needs to query the formula graph 
to find its dependents and calculate new formula results. 
Identifying dependents quickly is key to ensuring that 
users don't see stale or inconsistent results, and also helps spreadsheets return control early to users. 
Therefore, I proposed Taco to compress the formula graph by leveraging a key property, tabular locality, 
which means that cells close to each other are likely to have similar formula structures. 
Based on the analysis of real-world spreadsheets, 
I identified five tabular locality-based patterns 
and designed novel algorithms for querying the compressed 
representation without decompression. 
Our experiments on real-world spreadsheets show 
that the speedup of Taco over Excel on finding dependents 
is up to 632$\times$.}

\projectWithNumUrl{Modin: A Scalable Dataframe System}{Jan. 2021-Present}
{https://github.com/modin-project/modin}
{I led the effort in developing two key techniques 
for efficient parallel execution and metadata management in Modin.
Pandas is a popular dataframe library widely embraced by data scientists and has been 
the defacto tool for doing exploratory data analysis in Python. 
However, building a scalable dataframe system 
that maintains the unique semantics of dataframes 
(e.g., supporting mixed types of data in a column 
and requiring a specific row order for 
the output dataframe of a function) 
and supports a large number of pandas functions (i.e., over 600) 
is challenging. Modin addresses the challenges by mapping pandas functions 
to 15 core operators, and efficiently parallelizing 
the core operators and managing the associated metadata 
(e.g., the type information per column). 
Specifically, I formally developed decomposition rules 
that decompose the execution of a core operator 
in row-wise, column-wise, and cell-wise ways 
to parallelize the core operator 
while maintaining its unique semantics. 
I then developed rewriting rules to 
choose the decomposition rules for efficient parallel execution. 
For efficient metadata management, 
I proposed metadata independence, 
a technique that always logically maintains the metadata 
and lazily materializes the metadata when necessary. 
Modin has been adopted by many data scientists 
to accelerate their pandas execution. 
It has more than 7K daily downloads and 8K stars on GitHub.}

\projectWithNumUrl{Lux: Always-on Visualization Recommendations}{Jan. 2021-Present}
{https://github.com/lux-org/lux}
{Lux is a visualization recommendation tool to reduce the programming overhead for generating visualizations. 
It proposes a novel data-centered intent language that 
allows users to use attributes and filters to specify the portion of data of interest 
without having to consider the visualization encodings, which are inferred by Lux automatically. 
In addition, Lux models the recommendation process as a state machine, 
where the state space is defined by the possible combinations 
of attributes and filters and each recommendation moves users from the current state to an adjacent one. 
For example, if a user specifies the intent on two attributes and we consider the attribute space, 
Lux will visualize the relationship between the two attributes 
(e.g., using a scatter plot) and additionally consider the visualizations that remove, add, or swap one 
of the original two attributes as candidates. For the candidate visualizations to be recommended, 
Lux ranks them based on pre-defined interestingness scores and choosees the top $k$ visualizations, 
where $k$ is configurable. Similar recommendations can be applied to the filter space. 
Lux is integrated into the pandas dataframe workflow, where users visualize their dataframe 
by simply printing the dataframe. Lux has 400K downloads and 4.2K stars on Github, 
and is adopted in many domains, including medicine, education, finance, and more.}

\projectWithNum{Client-Assisted Data Loading}{July 2019-Dec. 2020}
{Data loading is time-consuming due to type parsing,  
integrity checking, and maintaining data structures. 
I mentored an undergraduate to leverage 
lazy data loading to make data quickly available 
without heavily sacrificing query performance. 
The idea is actively pushing predicates of prospective queries into 
the clients (e.g. edge devices) 
to generate bit-vectors that indicate whether a tuple 
is valid for a predicate. 
The system leverages the bit vectors 
to partially load data that is most frequently 
accessed by prospective queries. 
When queries access unloaded data, 
the system uses the bit-vectors as an index 
to accelerate the query execution by skipping irrelevant tuples.}

\projectWithNumUrl{\db: Resource-Efficient Database Execution}{Nov. 2017-Mar. 2021}
{https://github.com/orgs/crocodiledb/repositories}
{Scalable data systems, while performant, 
consume many computing and memory resources, 
introducing a high monetary cost if run on the cloud. 
I built a resource-efficient database, \db, 
to reduce resource usage while meeting a performance goal. 
% by re-architecting the query execution engine 
% and memory management in the database. 
\db supports scheduled queries 
over a stream of tuples (e.g., ETL jobs or recurring dashboard reports) 
and the performance goal is the maximally allowed query latency, 
which is defined as the time between when the full data 
is ready for one scheduled query (i.e., the last tuple arrives)
and the query result is computed. 
The core idea for reducing resource consumption 
is regarding the performance goal as a time slackness 
and designing novel system strategies 
to exploit the slackness, including 1) selectively 
executing parts of a scheduled query lazily, 
2) judiciously deciding the queries to share, and 
3) intelligently choosing the intermediate states 
(e.g., a hash table for a hash join) to maintain in memory.  
Parts of the techniques in \db are adopted 
in a system in Alibaba for reducing the resource 
consumption of recurring jobs for daily reports. 

Specifically, I developed \incability-aware query processing, 
or \inqp, for reducing the CPU consumption 
for one scheduled query. To reduce the query latency 
and meet the performance goal, 
incremental execution is employed to compute the partial results 
early and incrementally maintain the previous results. 
The more eagerly we incrementally execute a query (e.g., perform 
one incremental execution for each new tuple), 
the higher query work and subsequent CPU consumption 
there will be, mainly because output tuples in earlier executions 
may be removed by later executions, increasing the query work. 
Interestingly, eager incremental execution of different parts of a query 
will increase different amounts of query work for 
the same amount of reduced latency. 
Therefore, I proposed \incability, a novel metric for quantifying the 
cost-effectiveness of incremental execution 
for different parts of a query, 
and developed an algorithm for incrementally executing 
different sub-queries at different levels of eagerness 
based on their \incability 
to reduce total query work while meeting the performance goal. 

Next, I developed iShare, 
which selectively shares queries 
that execute on the same data and have different performance goals. 
Shared execution eliminates redundant computation to save CPU
cycles. However, shared execution for different performance goals 
requires the shared plan to meet the hardest goal (i.e., 
the lowest latency constraint) and forces some participating queries 
to run more eagerly. Eager incremental execution 
increases total query work, which may offset the benefits 
of shared execution. iShare considers the two factors
together and finds efficient query plans that reduce 
redundant work across concurrent queries and avoid the cost of 
eager incremental execution.

Finally, I developed intermittent query processing, 
or \iqp, for reducing memory consumption. 
Our observation is that the new data for a scheduled query 
may arrive intermittently, rather than continuously in some cases. 
Therefore, there may be a long time when the query does not  
incorporate new data. During this period, 
\iqp selectively discards parts of the intermediate 
states to reduce memory consumption with respect to a memory budget 
and recomputes them from other saved intermediate states when necessary. 
I proposed an algorithm to decide which intermediate states 
to discard to best reduce the query latency 
based on the predictive information about the new data, 
such as the estimated size and 
distribution of the base tables having new data.}

\projectWithNumUrl{Adaptive Concurrency Control for Main-Memory Databases}{Sep. 2015-Nov. 2017}
{https://github.com/dx-tang/cc-testbed}
{I built a main-memory database that adaptively mixes 
multiple concurrency control protocols, where each protocol is optimized for 
a different type of workload. I implemented multiple protocols, developed a machine learning model to 
predict the ideal protocol for a workload, and developed a mediated protocol for switching protocols online. 
Our experimental results show that this approach has much higher transaction throughput than 
the best single protocol under varied workloads. }

%\section{Earlier Projects (Before Ph.D.)}
%\projectWithNumIndent{Structured Data Shuffling for Big Data Analytical Stacks}{Nov. 2013-Jan. 2015}
%{We build a structured data shuffling procedure that can leverage the semantics of SQL queries to apply efficient compression algorithms 
%and discard unnecessary data during data shuffling.}
%
%\projectWithNumIndent{A Fast and Space-Efficient Join Method for Log Processing in MapReduce}{Sep. 2012-Nov. 2013}
%{We design a join method that achieves high query performance with a small extra storage cost for log processing. 
%It shuffles the log table to avoid huge storage consumption and optimizes the shuffle procedure to achieve high query performance.}

\vspace{-5mm}

\end{comment}

%------------------------------------------------

%----------------------------------------------------------------------------------------
%	REFEREE SECTION
%----------------------------------------------------------------------------------------

\begin{comment}
\section{Referees}

\parbox{0.5\textwidth}{ % First block
\begin{tabbing}
\hspace{2cm} \= \hspace{4cm} \= \kill % Spacing within the block
{\bf Name} \> Aditya G. Parameswaran \\ % Referee name
{\bf Affiliation} \> University of California, Berkeley \\ % Referee company
{\bf Position} \> Associate Professor \\ % Referee job title 
{\bf Contact} \> adityagp@eecs.berkeley.edu % Referee contact information
\end{tabbing}}
\\
\parbox{0.5\textwidth}{ % First block
\begin{tabbing}
\hspace{2cm} \= \hspace{4cm} \= \kill % Spacing within the block
{\bf Name} \> Aaron J. Elmore \\ % Referee name
{\bf Affiliation} \> University of Chicago \\ % Referee company
{\bf Position} \> Associate Professor \\ % Referee job title 
{\bf Contact} \> aelmore@cs.uchicago.edu % Referee contact information
\end{tabbing}}
\\
\parbox{0.5\textwidth}{ % Second block
\begin{tabbing}
\hspace{2cm} \= \hspace{4cm} \= \kill % Spacing within the block
{\bf Name} \> Michael J. Franklin \\ % Referee name
{\bf Affiliation} \> University of Chicago \\ % Referee company
{\bf Position} \> Full Professor, Liew Family Chairman of Computer Science \\ % Referee job title 
{\bf Contact} \> mjfranklin@uchicago.edu % Referee contact information
\end{tabbing}}
\\
\parbox{0.5\textwidth}{ % First block
\begin{tabbing}
\hspace{2cm} \= \hspace{4cm} \= \kill % Spacing within the block
{\bf Name} \> Sanjay Krishnan \\ % Referee name
{\bf Affiliation} \> University of Chicago \\ % Referee company
{\bf Position} \> Assistant Professor \\ % Referee job title 
{\bf Contact} \> skr@cs.uchicago.edu % Referee contact information
\end{tabbing}}
\\
\parbox{0.5\textwidth}{ % First block
\begin{tabbing}
\hspace{2cm} \= \hspace{4cm} \= \kill % Spacing within the block
{\bf Name} \> Indranil Gupta \\ % Referee name
{\bf Affiliation} \> University of Illinois Urbana-Champaign \\ % Referee company
{\bf Position} \> Full Professor \\ % Referee job title 
{\bf Contact} \> indy@illinois.edu % Referee contact information
\end{tabbing}}
\\
\parbox{0.5\textwidth}{ % Second block
\begin{tabbing}
\hspace{2cm} \= \hspace{4cm} \= \kill % Spacing within the block
{\bf Name} \> Umar Farooq Minhas \\ % Referee name
{\bf Affiliation} \> Apple Knowledge Platform \\ % Referee company
{\bf Position} \> Engineering Leader \\ % Referee job title 
{\bf Contact} \> umarfm13@gmail.com % Referee contact information
\end{tabbing}}
\\
\parbox{0.5\textwidth}{ % First block
\begin{tabbing}
\hspace{2cm} \= \hspace{4cm} \= \kill % Spacing within the block
{\bf Name} \> Wei Li\\ % Referee name
{\bf Affiliation} \> Institute of Computing Technology, Chinese Academy of Sciences\\ % Referee company
{\bf Position} \> Associate Professor \\ % Referee job title 
{\bf Contact} \> liwei@ict.ac.cn % Referee contact information
\end{tabbing}}
\hfill

\end{comment}

%----------------------------------------------------------------------------------------

\end{document}
